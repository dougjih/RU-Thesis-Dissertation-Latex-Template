\begin{my_abstract}

Determining whether two cells have the same genotype is a key problem in forensic deoxyribonucleic acid (DNA) analysis using electropherograms (EPGs). Single-cell extraction of DNA has become more practical in recent years and holds promise in reducing the complexity of downstream data interpretation. Clustering of EPG data is a potential way to aid downstream interpretation.

We explore many clustering techniques. The clustering methods that meet certain criteria are selected for further experiments with combinations of sample types and parameters including two proposed per-locus feature transformations. The best performing clusterers form cluster ensembles for further testing. Clustering performances are measured and compared to the baseline Mclust using $L^{1}$-norm. For EPGs from saliva samples, the best clusterer using affinity propagation scores 39\% higher in metrics mean and the best clusterer using mean-shift scores 528\% higher in perfect clustering for unfiltered data; whereas Mclust using $L^{2}$-norm scores 3\% higher in metrics mean and the best clusterer using HDBSCAN scores 43\% higher in perfect clustering for highpass-filtered data. For EPGs from blood samples, Mclust using $L^{1}$-norm is the best in metrics mean and Mclust using $L^{2}$-norm scores 2\% higher in perfect clustering for unfiltered data; and Mclust using $L^{1}$-norm is the best in metrics mean and in perfect clustering for highpass-filtered data. For EPGs from all samples, the best clusterer using BIRCH scores 1\% higher in metrics mean and Mclust using $L^{2}$-norm scores 4\% higher in perfect clustering for unfiltered data; and Mclust using $L^{1}$-norm is the best in metrics mean and in perfect clustering for highpass-filtered data.

Whereas various clusterers in the experiments perform better on saliva samples, the Mclust clusterers mostly perform better on blood samples and mixtures of both. Cluster ensembles as tested do not show better performances than individual clusterers, although the MCLA ensembles may be useful in providing consensus results in real use cases when which clusterer performs best is generally unknowable.

\end{my_abstract}